% Metódy inžinierskej práce

\documentclass{article}

\usepackage[slovak]{babel}
%\usepackage[T1]{fontenc}
\usepackage[IL2]{fontenc} % lepšia sadzba písmena Ľ než v T1
\usepackage[utf8]{inputenc}
\usepackage{graphicx}

\usepackage{url} % príkaz \url na formátovanie URL
\usepackage{hyperref} % odkazy v texte budú aktívne (pri niektorých triedach dokumentov spôsobuje posun textu)
\usepackage{cite}
%\usepackage{times}
\pagestyle{plain}

\title{Využívanie hier pri liečení ľudí s vážnym zranením
alebo ochorením\thanks{Semestrálny projekt v predmete Metódy inžinierskej práce, ak. rok 2015/16, vedenie: Ing. Fedor Lehocki, PhD.}} % meno a priezvisko vyučujúceho na cvičeniach

\author{Juraj Lančarič \\[2pt]
	{\small Slovenská technická univerzita v Bratislave}\\
	{\small Fakulta informatiky a informačných technológií}\\
	{\small \texttt{xlancaricj@stuba.sk}}
	}

\date{\small 6.11.2022} % upravte



\begin{document}

\maketitle

\begin{abstract}
Abstrakt-Hlavným účelom tohto článku bude ukázať, akým spôsom sa môžu videohry použiť na pomoc chorým a zraneným ľuďom. Taktiež sa budem snažiť odpovedať na otázku, či je podľa mňa vôbec možné používať videohry ako prostriedok na liečenie. Ukážem prípady kedy malo hranie videohier počas liečby na pacienta pozitívne účinky a taktiež kedy boli účinky negatívne. V článku poskytnem citácie doktorov a vedcov, ktorí podobnú možnosť skúmali spolu s ich názormi na danú problematiku. Ukážem ako videohry môžu pacientovi ulahčiť priebeh jeho uzdravenia a motivovať ho k tomu, aby sa nebál podstúpiť liečbu, či iné medikačné praktiky ako napríklad rehabilitáciu.
\end{abstract}



\section{Úvod}
Keď sa povie slovo videohry väčšina z nás si ako prvé nepredstavý práve ich využitie v medicine. Skôr si predstavý pravý opak, čiže zdravotné ťažkosti, ktoré tento koníček alebo práca prináša napríklad: obezita, epilepsia alebo aj závislosť. Veľa vedcov a výskumníkov sa ale v poslednej dobe snaží využiť pozitíva videohier, aby mohli pomôcť svojim pacientom, či už ide o pacientov so zdravotným postihnutím pacientov trpiacich vážnou chorobou alebo pacientov so zlým duševným zdravím. Odborníci na túto problematiku sa snažia aplikovať videohry do ich liečby. 
Za posledné obdobie vzniklo veľa výskumov kde sa vedci zaoberali touto témou, ale taktiež sa zaoberali či majú vôbec videohry v medicíne čo hľadať, či majú pozitívny a dlhodobý dopad na pacienta, ako ich najlepšie implementovať a taktiež či neexistujú ekefktívnejšie a lacnejšie spôsoby, ktoré by mohli v končenom dôsledku pomôcť pacientovi viac.
Asi najviac skúmanou hernou konzolou je konozola od spoločnosti Nintendo, Nintendo Wii. Budete sa tu môcť dočítať ako táto herná konzola pomáha pacientom s ich rehabilitáciou \cite{Simas:Videogames}. 
Taktiež tu budeme mať porovnanie s Kinectom od Microsoftu \cite{Perez:Approach}ktorý sa tiež ako Nintendo Wii začal používať pri rôznych rehabilitáciach a pri pomoci s postihnutými pacientmi.

%V úvode budem ešte pokračovať



\section{Nintendo Wii}
Nintendo Wii bola prvá konzola ktorú ste mohli ovládať aj pomocou pohybu a taktiež existujú na ňu mnoho hier ktoré musíte ovládať pomocou pohybu ako Wii Fit, alebo Wii Sports. Ešte väčší pohyb pri hraní umožňuje Wii Balance Board ktorá dokáže zaznamenať vašu rýchlosť a preniesť ju do hry. Kvôli týmto vlastnostiam sa stala adeptom na použitie ako nástroj pri rehabilitáciach. 
\includegraphics[scale=0.4]{fit}
\thanks{\cite{medsci10010013}}\\
Bol vedení výskum medzi tromi fyzioteraprutickými klinikami , dva na Azovských ostrovoch a jedna v Porte v Portugalsku \cite{Simas:Videogames}. Na týchto klinikách používali Nintendo Wii ako nástroj na rehbilitáciu.
%Tuto doplním ešte text a informácie
\\Podľa môjho názoru určite viac motivuje pacientov (hlavne mladších) keď berú rehabilitáciu ako hru, keďže ich to určite viac motivuje ako bežné cvičenie a iné rehabilitačné procedúry. Avšak ako bolo už povedané konzola Nintedo Wii nebola primárne vytvorená ako prostriedok na rehabilitáciu , takže aj celková práca a príprava aby sa vôbec táto konzola dala použiť na rehabilitáciu je viac náročná ako bežné cvičenie, kde človek nepotrebuje veľa. 
\section{Kinect}
Podobne ako u Nintenda Wii u Kinectu od Microsoftu je hlavný účel hranie pomocou pohybu. Kinect zaznamenáva pohyb osoby a pohyb následne prevádza do hry, a narozdiel od Nintedna Wii nemusíte mať v ruke žiadny ovládač, keďže váš pohyb sa zaznamenáva kamerou. 
\section{Porovnanie}



\section{Využitie pri pacientoch s rakovinou} 
\section{Zhrnutie}



% týmto sa generuje zoznam literatúry z obsahu súboru literatura.bib podľa toho, na čo sa v článku odkazujete
\bibliography{literatura}
\bibliographystyle{abbrv} % prípadne alpha, abbrv alebo hociktorý iný
\end{document}
